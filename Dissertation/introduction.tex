\chapter*{Введение}							% Заголовок
\addcontentsline{toc}{chapter}{Введение}	% Добавляем его в оглавление

\newcommand{\actuality}{}
\newcommand{\progress}{}
\newcommand{\aim}{{\textbf\aimTXT}}
\newcommand{\tasks}{\textbf{\tasksTXT}}
\newcommand{\novelty}{\textbf{\noveltyTXT}}
\newcommand{\influence}{\textbf{\influenceTXT}}
\newcommand{\methods}{\textbf{\methodsTXT}}
\newcommand{\defpositions}{\textbf{\defpositionsTXT}}
\newcommand{\reliability}{\textbf{\reliabilityTXT}}
\newcommand{\probation}{\textbf{\probationTXT}}
\newcommand{\contribution}{\textbf{\contributionTXT}}
\newcommand{\publications}{\textbf{\publicationsTXT}}


{\actuality} 
Ежедневное усложнение технических объектов управления (ОУ) и все большее внедрение средств автоматизации в различные сферы человеческой деятельности приводят к необходимости развития средств интеллектуального управления. Такие средства интеллектуального управления должны быть применимы в условиях параметрических и структурных неопределенностей. Классические методы теории автоматического управления, хоть и являются достаточно изученными и широко применимыми, имеют целый ряд неустранимых недостатков. К примеру, регуляторы с постоянными настройками не обеспечивают должного качества в условиях, когда параметры ОУ меняются во время функционирования или имеют значительный разброс для разных копий ОУ. К тому же, современные ОУ - зачастую нелинейные объекты с весьма приблизительным математическим описанием. Задача идентификации ОУ также бывает нетривиальной из-за наличия помех в процессе реального функционирования. Решение данных проблем связывают с развитием интеллектуальных систем управления в основе которых лежат искусственные нейронные сети.

С конца 1980-х начала 1990-х активное развитие получила группа алгоритмов обучения с подкреплением (англ. reinforcement learning - RL), которая является одним из разделов машинного обучения (англ. machine learning). После некоторого затишья, в 2013 году, учеными компании DeepMind был разработан модифицированный алгоритм обучения с подкреплением, в основе которого лежали глубокие искусственные нейронные сети, который показал блестящий результат в решении управлении игроком для игр Atari []. Затем, в 2016, учеными той же компании была разработана система на основе обучения с подкреплением для игры в Го, которая оказалась сильней лучших игроков на тот момент []. В 2017 году учеными некоммерческой компании OpenAI разработан бот, также на основе обучения с подкреплением, для игры в Dota 2 в модификации 1 на 1. Данный класс игр с неполной информации казался непосильным рубежом для алгоритмов искусственного интеллекта, однако этот бот обыграл лучшего игрока. В основе метода обучения с подкреплением лежит идея, заимствованная из живой природы, которая позволяет организмам приспосабливаться к изменяющимся условиям среды, а именно получение сигнала подкрепления, который свидетельствует о том, хорошо ли система функционирует. К примеру, хищник, который умеет хорошо охотится, получит подкрепление в виде пойманной жертвы, и наоборот, недостаточно "умная" жертва получит отрицательное подкрепление в виде ранения или смерти, если её догонит хищник. Агент, функционирующий по методу обучения с подкреплением получает сигнал подкрепления от среды, и, в зависимости от этого сигнала, меняет стратегию своего поведения.

Первой особенностью данного метода является то, что система управления не обладает никакой информации о среде и получает все данные только в результате обучения, в котором происходит взаимодействие со средой. Второй особенностью метода обучения с подкреплением является формирование управляющих воздействий с учетом информации о подкреплении, которое будет получено в будущем.


% {\progress} 
% Этот раздел должен быть отдельным структурным элементом по
% ГОСТ, но он, как правило, включается в описание актуальности
% темы. Нужен он отдельным структурынм элемементом или нет ---
% смотрите другие диссертации вашего совета, скорее всего не нужен.

{\aim} данной работы является разработка нейросетевого метода интеллектуального управления, базирующегося на принципах обучения с подкреплением и обеспечивающего возможность мультиагентного взаимодействия.

Для~достижения поставленной цели необходимо было решить следующие {\tasks}:
\begin{enumerate}
	\item Анализ нейросетевых методов управления техническими объектами
	\item Разработка структуры модели на основе нейросетевого обучения с подркеплением для автоматического управления техническими объектами
	\item Разработка програмного обеспечения, реализующего нейросетевой метод управления техническими объектами на основе обучения с подкреплением

  \item Разработка структурной схемы нейросетевой Актор-Критик САУ.
  \item Разработка структурной схемы кооперативного взаимодействия агентов.
  \item Разработка программных средств для функционирования нейросетевой Актор-Критик САУ и сред функционирования агентов.
  \item Выработка рекомендаций по настройке параметров обучения нейросетевой Актор-Критик САУ.
  \item Апробация разработанного метода управления в задачах управления нелинейными ОУ.
\end{enumerate}


{\novelty}
\begin{enumerate}
  \item Впервые получена структура нейросетевая Актор-Критик САУ на основе обучения с подкреплением, которая позволяет формировать управляющее воздействие на ОУ в соответствии с выбранным криетрием функционирования в условиях неизвестных или изменяющихся свойств ОУ.
  \item Впервые получена и применена Кооперативная нейросетевая Актор-Критик САУ на основе обучения с подкреплением, которая позволяет в процессе обучения самостоятельно выработать метод кооперации.
  \item Модифицированный алгоритм обучения нейронных сетей Актора и Критика, который обеспечивает устойчивость процесса обучения, а также быстроту сходимости.
  \item Применен нейросетевой Актор-Критик метод для управления погрузчиком на модели автоматизированного склада
\end{enumerate}

{\influence} Предлагаемый метод может быть использован при разработке интеллектуальных систем управления для ОУ без информации о структуре и параметрах математической модели.

{\methods} В работе были применены методы системного анализа, математического моделирования, теории управления, теории оптимизации, теории нейронных сетей, теории вероятности.

{\defpositions}
\begin{enumerate}
	\item Метод синтеза нейросетевой Актор-Критик САУ на основе обучения с подкреплением для ОУ с неизвестными или изменяющимися свойствами
	\item Нейросетевая Актор-Критик САУ способная управлять ОУ для выполнения одновременно нескольких задач
	\item Кооперативная нейросетевая Актор-Критик САУ на основе модифицированного алгоритма позволяет в процессе обучения самостоятельно выработать метод кооперации.
	\item Кооперативная нейросетевая Актор-Критик САУ в задаче управления несколькими погрузчиками на автоматическом складе
\end{enumerate}

{\reliability} полученных результатов обеспечивается \ldots \ Результаты находятся в соответствии с результатами, полученными другими авторами.


{\probation}
Основные результаты работы докладывались~на:
перечисление основных конференций, симпозиумов и~т.\:п.

{\contribution} Автором непосредственно получены все основные результаты работы: разработана структуры Актор-Критик САУ, Кооперативной Актор-Критик САУ, написаны программы <<Агент Актор-Критик>>, <<Агента Кооперативный Актор-Критик>>.

%\publications\ Основные результаты по теме диссертации изложены в ХХ печатных изданиях~\cite{Sokolov,Gaidaenko,Lermontov,Management},
%Х из которых изданы в журналах, рекомендованных ВАК~\cite{Sokolov,Gaidaenko}, 
%ХХ --- в тезисах докладов~\cite{Lermontov,Management}.

\ifnumequal{\value{bibliosel}}{0}{% Встроенная реализация с загрузкой файла через движок bibtex8
    \publications\ Основные результаты по теме диссертации изложены в XX печатных изданиях, 
    X из которых изданы в журналах, рекомендованных ВАК, 
    X "--- в тезисах докладов.%
}{% Реализация пакетом biblatex через движок biber
%Сделана отдельная секция, чтобы не отображались в списке цитированных материалов
    \begin{refsection}[vak,papers,conf]% Подсчет и нумерация авторских работ. Засчитываются только те, которые были прописаны внутри \nocite{}.
        %Чтобы сменить порядок разделов в сгрупированном списке литературы необходимо перетасовать следующие три строчки, а также команды в разделе \newcommand*{\insertbiblioauthorgrouped} в файле biblio/biblatex.tex
        \printbibliography[heading=countauthorvak, env=countauthorvak, keyword=biblioauthorvak, section=1]%
        \printbibliography[heading=countauthorconf, env=countauthorconf, keyword=biblioauthorconf, section=1]%
        \printbibliography[heading=countauthornotvak, env=countauthornotvak, keyword=biblioauthornotvak, section=1]%
        \printbibliography[heading=countauthor, env=countauthor, keyword=biblioauthor, section=1]%
        \nocite{%Порядок перечисления в этом блоке определяет порядок вывода в списке публикаций автора
                vakbib1,vakbib2,%
                confbib1,confbib2,%
                bib1,bib2,%
        }%
        \publications\ Основные результаты по теме диссертации изложены в \arabic{citeauthor} печатных изданиях, 
        \arabic{citeauthorvak} из которых изданы в журналах, рекомендованных ВАК, 
        \arabic{citeauthorconf} "--- в тезисах докладов.
    \end{refsection}
    \begin{refsection}[vak,papers,conf]%Блок, позволяющий отобрать из всех работ автора наиболее значимые, и только их вывести в автореферате, но считать в блоке выше общее число работ
        \printbibliography[heading=countauthorvak, env=countauthorvak, keyword=biblioauthorvak, section=2]%
        \printbibliography[heading=countauthornotvak, env=countauthornotvak, keyword=biblioauthornotvak, section=2]%
        \printbibliography[heading=countauthorconf, env=countauthorconf, keyword=biblioauthorconf, section=2]%
        \printbibliography[heading=countauthor, env=countauthor, keyword=biblioauthor, section=2]%
        \nocite{vakbib2}%vak
        \nocite{bib1}%notvak
        \nocite{confbib1}%conf
    \end{refsection}
}
При использовании пакета \verb!biblatex! для автоматического подсчёта
количества публикаций автора по теме диссертации, необходимо
их здесь перечислить с использованием команды \verb!\nocite!.
    

 % Характеристика работы по структуре во введении и в автореферате не отличается (ГОСТ Р 7.0.11, пункты 5.3.1 и 9.2.1), потому её загружаем из одного и того же внешнего файла, предварительно задав форму выделения некоторым параметрам

\textbf{Объем и структура работы.} Диссертация состоит из~введения, трёх глав, заключения и~двух приложений.
%% на случай ошибок оставляю исходный кусок на месте, закомментированным
%Полный объём диссертации составляет  \ref*{TotPages}~страницу с~\totalfigures{}~рисунками и~\totaltables{}~таблицами. Список литературы содержит \total{citenum}~наименований.
%
Полный объём диссертации составляет
\formbytotal{TotPages}{страниц}{у}{ы}{}, включая
\formbytotal{totalcount@figure}{рисун}{ок}{ка}{ков} и
\formbytotal{totalcount@table}{таблиц}{у}{ы}{}.   Список литературы содержит  
\formbytotal{citenum}{наименован}{ие}{ия}{ий}.
