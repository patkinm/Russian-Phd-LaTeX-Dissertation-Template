\chapter*{Список сокращений и условных обозначений}             % Заголовок
\addcontentsline{toc}{chapter}{Список сокращений и условных обозначений}  % Добавляем его в оглавление
\noindent
\addtocounter{table}{-1}% Нужно откатить на единицу счетчик номеров таблиц, так как следующая таблица сделана для удобства представления информации по ГОСТ
%\begin{longtabu} to \dimexpr \textwidth-5\tabcolsep {r X}
\begin{longtabu} to \textwidth {r X}
% Жирное начертание для математических символов может иметь
% дополнительный смысл, поэтому они приводятся как в тексте
% диссертации


\textbf{ИНС} & искусственная нейронная сеть\\
\textbf{ОУ} & объект управления\\
\textbf{САУ} & система автоматического управления\\
\textbf{УУ} & управляющее устройство\\
\textbf{АКМ} & Актор-Критик метод\\
\textbf{RL} & reinforcement learning, обучение с подкреплением\\

\end{longtabu}
