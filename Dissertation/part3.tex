\chapter{Нейросетевой метод управления на основе Актор-Критик обучения с подкреплением} \label{chapt3}

\section{Классификация ИНС} \label{sect3_1}

\subsection{Сети прямого распространения} \label{subsect3_1_1}

\subsection{Рекуррентные сети} \label{subsect3_1_2}

\section{Программное средство} \label{sect3_2}

\section{Экспериментальные исследования} \label{sect3_3}

\subsection{Управление машиной Дубинса в задаче преследования} \label{subsect3_3_1}

\begin{equation}
\label{eq:3_3_1p1}
\left\{
\begin{alignedat}{2}
\frac{dx}{dt}=v\cdot \cos (\phi ) \\
\frac{dy}{dt}=v\cdot \sin (\phi ) \\
\frac{d\phi}{dt}=\frac{\pi}{2}(u_{rm}-u_{lm}) \\
v=0.5(u_{rm}-u_{lm})
\end{alignedat}
\right.
\text{где}
\left\{
\begin{alignedat}{2}
u_{rm}=[0,1]\\
u_{lm}=[0,1]
\end{alignedat}
\right.
\end{equation}

\subsection{Мультиагентное взаимодействие в задаче преследования} \label{subsect3_3_2}

\subsection{Управление настройками ПИ-регулятора в задаче управления двуосным прокатным станом} \label{subsect3_3_3}

\clearpage
\clearpage
