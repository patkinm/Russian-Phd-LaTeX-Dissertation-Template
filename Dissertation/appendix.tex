\appendix
%%% Оформление заголовков приложений ближе к ГОСТ:
\setlength{\midchapskip}{20pt}
\renewcommand*{\afterchapternum}{\par\nobreak\vskip \midchapskip}
\renewcommand\thechapter{\Asbuk{chapter}} % Чтобы приложения русскими буквами нумеровались
   % Предварительные настройки для правильного подключения Приложений
\chapter{Листинг кода} \label{AppendixA}

\begin{python}
from keras.models import Sequential
from keras.layers.core import Dense, Dropout, Activation, Flatten
from keras.optimizers import RMSprop, SGD
import numpy as np
import random
from keras.utils.visualize_util import plot

VERBOSE = 0

class ac_agent():
def __init__(self, action_space, observation_space, gamma=0.9, epsilon=0.7, buffer_len=1024, actor_w=None, critic_w=None):
self.action_space = action_space
self.observation_space = observation_space
# self.actor_model = self.init_actor_model('/home/mikhail/projects/hockey_ai/ai_models/agent1112_actor_6.h5')
# self.critic_model = self.init_critic_model('/home/mikhail/projects/hockey_ai/ai_models/agent1112_critic_6.h5')
print(actor_w, critic_w)
self.actor_model = self.init_actor_model(weights=actor_w)
self.critic_model = self.init_critic_model(weights=critic_w)
self.actor_replay = []
self.critic_replay = []
self.gamma = gamma
self.batchSize = 32
self.buffer_len = buffer_len
self.episode_move_counter = 0
self.wins = 0
self.losses = 0
self.epsilon = epsilon
self.actor_hist = 0
self.critic_hist = 0

def init_actor_model(self, weights=None):
actor_model = Sequential()

actor_model.add(Dense(256, init='lecun_uniform', input_shape=(self.observation_space,)))
actor_model.add(Dropout(0.2))
actor_model.add(Activation('relu'))

actor_model.add(Dense(256, init='lecun_uniform'))
actor_model.add(Dropout(0.2))
actor_model.add(Activation('relu'))

# actor_model.add(Dense(256, init='lecun_uniform'))
# actor_model.add(Dropout(0.2))
# actor_model.add(Activation('relu'))


actor_model.add(Dense(self.action_space, init='lecun_uniform'))
actor_model.add(Activation('linear'))

if weights != None:
actor_model.load_weights(weights)

#a_optimizer = SGD(lr=0.01, decay=1e-6, momentum=0.9, nesterov=True)
a_optimizer = RMSprop()
#actor_model.summary()
actor_model.compile(loss='mse', optimizer=a_optimizer)
plot(actor_model, to_file='graphics/actor_model.png', show_shapes=True)
return actor_model

def init_critic_model(self, weights=None):
critic_model = Sequential()

critic_model.add(Dense(256, init='lecun_uniform', input_shape=(self.observation_space,)))
critic_model.add(Dropout(0.2))
critic_model.add(Activation('relu'))

critic_model.add(Dense(256, init='lecun_uniform'))
critic_model.add(Dropout(0.2))
critic_model.add(Activation('relu'))

# critic_model.add(Dense(256, init='lecun_uniform'))
# critic_model.add(Dropout(0.2))
# critic_model.add(Activation('relu'))

critic_model.add(Dense(1, init='lecun_uniform'))
critic_model.add(Activation('linear'))

if weights != None:
critic_model.load_weights(weights)

#c_optimizer = SGD(lr=0.01, decay=1e-6, momentum=0.9, nesterov=True)
c_optimizer = RMSprop()
critic_model.compile(loss='mse', optimizer=c_optimizer)
plot(critic_model, to_file='graphics/critic_model.png', show_shapes=True)
return critic_model

def predict(self, orig_state, epsilon):

if (random.random() < epsilon): #choose random action
action = np.random.randint(0, self.action_space)
else: #choose best action from Q(s,a) values
qval = self.actor_model.predict(orig_state.reshape(1,self.observation_space) )[0]
action = (np.argmax(qval))

return action
def train(self, new_observation, new_reward, new_action, done, orig_state, orig_reward):
orig_val = self.critic_model.predict(orig_state.reshape(1,self.observation_space))

new_state = np.array(new_observation)
new_val = self.critic_model.predict(new_state.reshape(1,self.observation_space))
if not done: # Non-terminal state.
target = orig_reward + ( self.gamma * new_val)
else:
# In terminal states, the environment tells us
# the value directly.
target = orig_reward + ( self.gamma * new_reward )
best_val = max((orig_val * self.gamma), target)
self.critic_replay.append([orig_state, best_val])
# If we are in a terminal state, append a replay for it also.
if done:
self.critic_replay.append( [new_state, float(new_reward)] )

actor_delta = new_val - orig_val
self.actor_replay.append([orig_state, new_action, actor_delta])

# Critic Replays...
while(len(self.critic_replay) > self.buffer_len): # Trim replay buffer
self.critic_replay.pop(0)

# Start training when we have enough samples.
if(len(self.critic_replay) >= self.buffer_len):
#for zz in range(100):
minibatch = random.sample(self.critic_replay, self.batchSize)
X_train = []
y_train = []
for memory in minibatch:
m_state, m_value = memory
y = np.empty([1])
y[0] = m_value
X_train.append(m_state.reshape((self.observation_space,)))
y_train.append(y.reshape((1,)))
X_train = np.array(X_train)
y_train = np.array(y_train)
if VERBOSE > 0:
print('critic_model')
self.critic_hist = (self.critic_hist + self.critic_model.fit(X_train, y_train, batch_size=self.batchSize, nb_epoch=1, verbose=VERBOSE, shuffle=False).history['loss'][0]) / 2

while(len(self.actor_replay) > self.buffer_len):
self.actor_replay.pop(0)

if(len(self.actor_replay) >= self.buffer_len):
#for zz in range(100):
X_train = []
y_train = []
minibatch = random.sample(self.actor_replay, self.batchSize)
for memory in minibatch:
m_orig_state, m_action, m_value = memory

old_qval = self.actor_model.predict( m_orig_state.reshape(1,self.observation_space,) )
y = np.zeros(( 1, self.action_space ))
y[:] = old_qval[:]

y[0][m_action] = m_value
#y[0] = np.array(m_value)
X_train.append(m_orig_state.reshape((self.observation_space,)))
y_train.append(y.reshape((self.action_space,)))
X_train = np.array(X_train)
y_train = np.array(y_train)
if VERBOSE > 0:
print('actor_model')
self.actor_hist = (self.actor_hist + self.actor_model.fit(X_train, y_train, batch_size=self.batchSize, nb_epoch=1, verbose=VERBOSE, shuffle=False).history['loss'][0]) / 2


# Bookkeeping at the end of the turn.
# observation = new_observation
# reward = new_reward
def save_state(self, prefix=''):
self.critic_model.save_weights('ai_models/' + prefix + 'critic_car.h5')
self.actor_model.save_weights('ai_models/'  + prefix + 'actor_car.h5')
\end{python}

\chapter{Листинг кода} \label{AppendixB}
\begin{python}
import random
import math
from math import *
import numpy as np
random.seed(1)
import pygame
from pygame.color import THECOLORS

import pymunk
from pymunk.vec2d import Vec2d
from pymunk.pygame_util import draw

# PyGame init
width = 1000
height = 700
pygame.init()
screen = pygame.display.set_mode((width, height))
clock = pygame.time.Clock()

max_dist = sqrt(width**2 + height**2)
# Turn off alpha since we don't use it.
screen.set_alpha(None)

# Showing sensors and redrawing slows things down.
show_sensors = True
draw_screen = True


def calc_dist(x1, y1, x2, y2):
return sqrt((x1-x2)**2 + (y1-y2)**2)

def get_angle_to_goal(x, y, angle, goal_x, goal_y):
absolute_angle_to = atan2(goal_y - y, goal_x - x)
relative_angle_to = absolute_angle_to - angle
while relative_angle_to > pi:
relative_angle_to -= 2.0 * pi

while relative_angle_to < -pi:
relative_angle_to += 2.0 * pi

return relative_angle_to

class GameState:
def __init__(self):

#angle_step = 1.0463 / random.uniform(0.5, 4)
#angle_step = 1.0463 / random.randint(1, 2)
self.angle_step = 1.0463 / 2
# Global-ish.
self.crashed = False

# Physics stuff.
self.space = pymunk.Space()
self.space.gravity = pymunk.Vec2d(0., 0.)

# Create the car.
self.create_car(100, 100, 0.5)

# Record steps.
self.num_steps = 0

# Create walls.
static = [
pymunk.Segment(
self.space.static_body,
(0, 1), (0, height), 1),
pymunk.Segment(
self.space.static_body,
(1, height), (width, height), 1),
pymunk.Segment(
self.space.static_body,
(width-1, height), (width-1, 1), 1),
pymunk.Segment(
self.space.static_body,
(1, 1), (width, 1), 1)
]
for s in static:
s.friction = 1.
s.group = 1
s.collision_type = 1
s.color = THECOLORS['red']
self.space.add(static)

# Create some obstacles, semi-randomly.
# We'll create three and they'll move around to prevent over-fitting.
self.obstacles = []
self.obstacles.append(self.create_obstacle(400, 450, 80))
self.obstacles.append(self.create_obstacle(600, 200, 80))
#self.obstacles.append(self.create_obstacle(600, 600, 35))

# self.goal_coord = (800, 600)
# self.goal_radius = 50
# self.create_goal(self.goal_coord[0], self.goal_coord[1], self.goal_radius)
self.prev_goal_dist = 999
self.steps = 0
self.max_steps = 100
# Create a cat.
self.create_cat()

def reset(self):
self.car_body.position = (200, 200)
self.car_body.angle = (random.random() - 0.5) * 1.6 * 2
#self.car_body.angle = -1.6

self.cat_body.position = 800, 600
self.cat_body.velocity = Vec2d(1, 0)*0
self.steps = 0


return self.get_features()[0]

def get_features(self):
x, y = self.car_body.position

features = []
goal_angle = get_angle_to_goal(x, y, self.car_body.angle, self.cat_body.position[0], self.cat_body.position[1])
readings = self.get_sonar_readings(x, y, self.car_body.angle)
goal_angle /= (2 * pi)
features.append(goal_angle+0.5)
for r in readings:
features.append(r / 40)
features.append(self.angle_step)
return features, readings

def create_obstacle(self, x, y, r):
c_body = pymunk.Body(pymunk.inf, pymunk.inf)
c_shape = pymunk.Circle(c_body, r)
c_shape.elasticity = 1.0
c_body.position = x, y
c_shape.color = THECOLORS["blue"]
self.space.add(c_body, c_shape)
return c_body

def create_goal(self, x, y, r):
c_body = pymunk.Body(pymunk.inf, pymunk.inf)
c_shape = pymunk.Circle(c_body, r)
c_shape.elasticity = 1.0
c_body.position = x, y
c_shape.color = THECOLORS["white"]
self.space.add(c_body, c_shape)
return c_body

def create_cat(self):
inertia = pymunk.moment_for_circle(1, 0, 14, (0, 0))
self.cat_body = pymunk.Body(1, inertia)
self.cat_body = pymunk.Body(pymunk.inf, pymunk.inf)
self.cat_body.position = 800, 600
self.cat_shape = pymunk.Circle(self.cat_body, 50)
self.cat_shape.color = THECOLORS["white"]
self.cat_shape.elasticity = 1.0
self.cat_shape.angle = 0.5
direction = Vec2d(1, 0).rotated(self.cat_body.angle)
self.space.add(self.cat_body, self.cat_shape)

def create_car(self, x, y, r):
inertia = pymunk.moment_for_circle(1, 0, 14, (0, 0))
self.car_body = pymunk.Body(1, inertia)
self.car_body.position = x, y
self.car_shape = pymunk.Circle(self.car_body, 25)
self.car_shape.color = THECOLORS["green"]
self.car_shape.elasticity = 0.0
self.car_body.angle = r
driving_direction = Vec2d(1, 0).rotated(self.car_body.angle)
self.car_body.apply_impulse(driving_direction)
self.space.add(self.car_body, self.car_shape)

def step(self, action):

if action == 0:  # Turn left.
self.car_body.angle -= self.angle_step
elif action == 1:  # Turn right.
self.car_body.angle += self.angle_step
elif action == 2:
pass

# Move obstacles.
# if self.num_steps % 100 == 0:
#     self.move_obstacles()

# # Move cat.
# if self.num_steps % 5 == 0:
#     self.move_cat()

driving_direction = Vec2d(1, 0).rotated(self.car_body.angle)
self.car_body.velocity = 100 * driving_direction
# self.car_body.velocity_limit = 20
# self.car_body.force = 0.5 * Vec2d(1, 0).rotated(self.car_body.angle)

# Update the screen and stuff.
screen.fill(THECOLORS["black"])
draw(screen, self.space)
self.space.step(0.5)
if draw_screen:
pygame.display.flip()
clock.tick()

# Get the current location and the readings there.
x, y = self.car_body.position
goal_distance = calc_dist(x, y, self.cat_body.position[0], self.cat_body.position[1])
features, readings = self.get_features()

# state = np.array([readings])
#
# # Set the reward.
# # Car crashed when any reading == 1
# if self.car_is_crashed(readings):
#     self.crashed = True
#     reward = -500
#     self.recover_from_crash(driving_direction)
# else:
#     # Higher readings are better, so return the sum.
#     reward = -5 + int(self.sum_readings(readings) / 10)
# self.num_steps += 1
timing_rate = (self.max_steps - self.steps) / self.max_steps
close_rate = (max_dist - goal_distance) / max_dist
#print(timing_rate, close_rate)
# assert timing_rate <= 1 and timing_rate >= 0
# assert close_rate <= 1 and close_rate >= 0
r1 = (timing_rate + close_rate) / 2 * 20

if self.prev_goal_dist > goal_distance:
reward = 10
else:
reward = -10


#reward -= self.steps * 0.09
self.prev_goal_dist = goal_distance
done = False
if self.car_is_crashed(readings):
reward = -50
done = True
if goal_distance < 100:
#add = (self.max_steps - self.steps) / 20
reward = 50
done = True

if self.steps > self.max_steps:
done = True
self.steps += 1

return features, reward, done

def move_obstacles(self):
# Randomly move obstacles around.
for obstacle in self.obstacles:
speed = random.randint(1, 5)
direction = Vec2d(1, 0).rotated(self.car_body.angle + random.randint(-2, 2))
obstacle.velocity = speed * direction

def move_cat(self):
speed = random.randint(50, 100)
self.cat_body.angle -= random.randint(-1, 1)
direction = Vec2d(1, 0).rotated(self.cat_body.angle)
self.cat_body.velocity = speed * direction

def car_is_crashed(self, readings):
if readings[0] == 1 or readings[1] == 1 or readings[2] == 1:
return True
else:
return False

def recover_from_crash(self, driving_direction):
"""
We hit something, so recover.
"""
while self.crashed:
# Go backwards.
self.car_body.velocity = -100 * driving_direction
self.crashed = False
for i in range(10):
self.car_body.angle += .2  # Turn a little.
screen.fill(THECOLORS["red"])  # Red is scary!
draw(screen, self.space)
self.space.step(1./10)
if draw_screen:
pygame.display.flip()
clock.tick()

def sum_readings(self, readings):
"""Sum the number of non-zero readings."""
tot = 0
for i in readings:
tot += i
return tot

def get_sonar_readings(self, x, y, angle):
readings = []
"""
Instead of using a grid of boolean(ish) sensors, sonar readings
simply return N "distance" readings, one for each sonar
we're simulating. The distance is a count of the first non-zero
reading starting at the object. For instance, if the fifth sensor
in a sonar "arm" is non-zero, then that arm returns a distance of 5.
"""
# Make our arms.
arm_left = self.make_sonar_arm(x, y)
arm_middle = arm_left
arm_right = arm_left

# Rotate them and get readings.
readings.append(self.get_arm_distance(arm_left, x, y, angle, 0.75))
readings.append(self.get_arm_distance(arm_middle, x, y, angle, 0))
readings.append(self.get_arm_distance(arm_right, x, y, angle, -0.75))

if show_sensors:
pygame.display.update()

return readings

def get_arm_distance(self, arm, x, y, angle, offset):
# Used to count the distance.
i = 0

# Look at each point and see if we've hit something.
for point in arm:
i += 1

# Move the point to the right spot.
rotated_p = self.get_rotated_point(
x, y, point[0], point[1], angle + offset
)

# Check if we've hit something. Return the current i (distance)
# if we did.
if rotated_p[0] <= 0 or rotated_p[1] <= 0 \
or rotated_p[0] >= width or rotated_p[1] >= height:
return i  # Sensor is off the screen.
else:
obs = screen.get_at(rotated_p)
if self.get_track_or_not(obs) != 0:
return i

if show_sensors:
pygame.draw.circle(screen, (255, 255, 255), (rotated_p), 2)

# Return the distance for the arm.
return i

def make_sonar_arm(self, x, y):
spread = 10  # Default spread.
distance = 20  # Gap before first sensor.
arm_points = []
# Make an arm. We build it flat because we'll rotate it about the
# center later.
for i in range(1, 40):
arm_points.append((distance + x + (spread * i), y))

return arm_points

def get_rotated_point(self, x_1, y_1, x_2, y_2, radians):
# Rotate x_2, y_2 around x_1, y_1 by angle.
x_change = (x_2 - x_1) * math.cos(radians) + \
(y_2 - y_1) * math.sin(radians)
y_change = (y_1 - y_2) * math.cos(radians) - \
(x_1 - x_2) * math.sin(radians)
new_x = x_change + x_1
new_y = height - (y_change + y_1)
return int(new_x), int(new_y)

def get_track_or_not(self, reading):
if reading == THECOLORS['black'] or reading == THECOLORS['white']:
return 0
else:
return 1

if __name__ == "__main__":
game_state = GameState()
c = 0
game_state.reset()
x_min = 1000
x_max = 0
while True:
#game_state.step((random.randint(0, 1)))
features, reward, done = game_state.step(0)
x_min = min(x_min, game_state.car_body.position[0])
x_max = max(x_max, game_state.car_body.position[0])
if done:
print('radius:', (x_max - x_min) / 2)
\end{python}